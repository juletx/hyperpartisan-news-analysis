%
% File acl2020.tex
%
%% Based on the style files for ACL 2020, which were
%% Based on the style files for ACL 2018, NAACL 2018/19, which were
%% Based on the style files for ACL-2015, with some improvements
%%  taken from the NAACL-2016 style
%% Based on the style files for ACL-2014, which were, in turn,
%% based on ACL-2013, ACL-2012, ACL-2011, ACL-2010, ACL-IJCNLP-2009,
%% EACL-2009, IJCNLP-2008...
%% Based on the style files for EACL 2006 by 
%%e.agirre@ehu.es or Sergi.Balari@uab.es
%% and that of ACL 08 by Joakim Nivre and Noah Smith

\documentclass[11pt,a4paper]{article}
\usepackage[hyperref]{acl2020}
\usepackage{times}
\usepackage{latexsym}
\renewcommand{\UrlFont}{\ttfamily\small}

% This is not strictly necessary, and may be commented out,
% but it will improve the layout of the manuscript,
% and will typically save some space.
\usepackage{microtype}

\aclfinalcopy % Uncomment this line for the final submission
%\def\aclpaperid{***} %  Enter the acl Paper ID here

%\setlength\titlebox{5cm}
% You can expand the titlebox if you need extra space
% to show all the authors. Please do not make the titlebox
% smaller than 5cm (the original size); we will check this
% in the camera-ready version and ask you to change it back

\usepackage{booktabs}
\usepackage{graphicx}

\newcommand\BibTeX{B\textsc{ib}\TeX}

\title{Hyperpartisan News Analysis With Scattertext}

\author{Julen Etxaniz \\
  University of the Basque Country \\
  \texttt{jetxaniz007@ikasle.ehu.eus} \\\And
  Oihane Cantero \\
  University of the Basque Country \\
  \texttt{ocantero003@ikasle.ehu.eus} \\}

\date{}

\begin{document}
\maketitle
\begin{abstract}

\end{abstract}

\section{Introduction}

Hyperpartisan news are those that take an extreme left-wing or right-wing standpoint. Detecting hyperpartisan news automatically can be useful to tag them and inform readers. This was the goal of the SemEval 2019 Task 4 \cite{kiesel2019semeval}.

The purpose of this work is to analyze the usage of words in documents which are hyperpartisan and non-hyperpartisan. Hyperpartisan news are those that exhibit blind, prejudiced, or unreasoning allegiance to one party, faction, cause, or person.

Whereas the task on semeval was to design a system to auto- matically detect hyperpartisan news, in this exercise we are going to exploit both corpora (hyperpartisan and not hyperpartisan news), and analyze which terms are the most relevant in each of the sets.

\section{Related Work}

\section{Dataset}

The data is split into multiple files. The articles are contained in the files with names starting with "articles-" (which validate against the XML schema article.xsd). The ground-truth information is contained in the files with names starting with "ground-truth-" (which validate against the XML schema ground-truth.xsd).

The dataset can be downloaded from Zenodo\footnote{\url{https://zenodo.org/record/5776081}}. You can also use the dataset creation script to create a HuggingFace Dataset automatically.

\subsection{By Publisher Dataset}

The first part of the data (filename contains "bypublisher") is labeled by the overall bias of the publisher as provided by BuzzFeed journalists or \href{https://mediabiasfactcheck.com}{MediaBiasFactCheck.com}. It contains a total of 750,000 articles, half of which (375,000) are hyperpartisan and half of which are not. Half of the articles that are hyperpartisan (187,500) are on the left side of the political spectrum, half are on the right side. This data is split into a training set (80\%, 600,000 articles) and a validation set (20\%, 150,000 articles), where no publisher that occurs in the training set also occurs in the validation set. Similarly, none of the publishers in those sets occurs in the test set (4,000 articles).

\subsection{By Article Dataset}

The second part of the data (filename contains "byarticle") is labeled through crowdsourcing on an article basis. The data contains only articles for which a consensus among the crowdsourcing workers existed. It contains a total of 645 articles. Of these, 238 (37\%) are hyperpartisan and 407 (63\%) are not, We will use a similar (but balanced!) test set that contains 628 articles. Again, none of the publishers in this set occurs in the test set.

\section{Methods}

First, we preprocess the original data to get better results in the analysis. Then, we use two different methods for analysing hyperpartisan and non-hyperpartisan documents. On the one hand, we calculate log-odd ratios to extract the most relevant words of each category. On the other hand, we use scattertext \cite{kessler2017scattertext} to build an interactive HTML scatter plot. Code is available at GitHub: \footnote{\url{https://github.com/juletx/hyperpartisan-news-detection}}

\subsection{Preprocessing}

As the original files are XML files, we have to preprocess them in order to obtain good insights. First we use the \texttt{lxml} library in python to analyze the XML documents and extract the necessary information. Preprocessing also includes tokenizing, converting words to lowercase, removing punctuation, numbers, stop words, XML entities and image tags.

To calculate the log-odd ratios, we select the validation set of the By Publisher dataset that contains 150,000 articles. The first step is to generate two text files for hyperpartisan and non-hyperpartisan news articles, respectively. You have to divide the News articles contained in \href{https://zenodo.org/record/5776081/files/articles-validation-bypublisher-20181122.zip?download=1}{articles-validation-bypublisher-20181122.xml.zip} into two text files (\texttt{hyperpartisan.txt} and \texttt{non-hyperpartisan.txt}), according to their ground truth value in \href{https://zenodo.org/record/5776081/files/ground-truth-validation-bypublisher-20181122.zip?download=1}{ground-truth-validation-bypublisher-20181122.xml.zip}.

For scattertext, we select the test set of the By Article Dataset, which contains 628 articles. Scattertext needs a smaller number of articles because otherwise the interactive site takes very long to load. This size is big enough to extract the most relevant words.

\subsection{Log-odd ratio}

After preprocessing text files, we extract the log-odd ratios of each word. Because the log-odd ratio is sensitive to infrequent words, we discard words that appear less than 20 times in the corpus. We also extract the log-odd ratios of the bigrams in the corpus. Having the log-odd ratios, we can extract the most relevant 50 words and bigrams in hyperpartisan and non-hyperpartisan documents. If we analyze these words, we can draw some conclusions about hyperpartisan news.

The log-odd ratio is a measure of words compared on two sets of documents ($i$ and $j$), which in our case corresponds to hyperpartisan and non-hyperpartisan documents, respectively. Each word then can be associated with its log-odd ratio $r_w$, which is a number that can be positive or negative: positive numbers are associated with set $i$, and negative numbers with set $j$.

The log-odd ratio $r_w$ is defined as:
$$p_w^{(i)} = \frac{f_w^{(i)}}{N^{(i)}}; p_w^{(j)} = \frac{f_w^{(j)}}{N^{(j)}}$$
$$o_w^{(i)} = \frac{p_w^{(i)}}{1-p_w^{(i)}}; o_w^{(j)} = \frac{p_w^{(j)}}{1-p_w^{(j)}}$$
$$r_w = \log{o_w^{(i)}} - \log{o_w^{(j)}}$$

where $f_w^{(i)}$ is the frequency of word $w$ in group $i$ (hyperpartisan or non-hyperpartisan), and $N^{(i)}$ is the number of words in group $i$.

For example, suppose that the word $gold$ appears $2,500$ times on hyperpartisan documents ($f_{gold}^{(i)} = 2500$), and $760$ times on non-hyperpartisan documents ($f_{gold}^{(j)} = 760$). Furthermore, suppose that there are $25,000$ words in hyperpartisan documents ($N^{i} = 25000$), and $17,500$ on non-hyperpartisan documents ($N^{j} = 17000$). Then:
$$p_{gold}^{(i)} = \frac{2500}{25000} = 0.1; p_{gold}^{(j)} = \frac{760}{17500} = 0.045$$
$$o_{gold}^{(i)} = \frac{0.1}{1-0.1} = 0.11; o_{gold}^{(j)} = \frac{0.045}{1-0.045} = 0.047$$
$$r_{gold} = \log{0.11} - \log{0.047} = 0.369$$

and therefore the log odd ratio of $gold$ is $0.369$.

\subsection{Scattertext}

Scattertext \cite{kessler2017scattertext} is a tool for finding distinguishing terms in corpora and displaying them in an interactive HTML scatter plot. It is intended for visualizing what words and phrases are more characteristic of a category than others. We can use it to compare hyperpartisan and non-hyperpartisan news. It could also be used to compare news with left and right bias.

\section{Results}

\subsection{Relevant Words}

There are many differences betweeen the 50 most relevant words of
hyperpartisan and non-hyperpartisan news. Here are the main findings of
each class. Table \ref{tab:words_bigrams} shows all words and bigrams.

\begin{table*}[ht]
\centering
\begin{tabular}{l|l|ll|ll}
\toprule
Hyp Words        & Non Words         & \multicolumn{2}{c}{Hyp Bigrams}  & \multicolumn{2}{c}{Non Bigrams}   \\ \midrule
wonkette          & subsaharan        & repeat      & offenders   & texas         & tribune      \\
vulgarity         & straus            & state       & shall       & may           & subject      \\
realclearpolitics & treasuries        & media       & keep        & emerging      & economies    \\
newsbusters       & zuma              & reserve     & right       & complete      & list         \\
oreilly           & boko              & media       & research    & exchange      & rate         \\
profanity         & tic               & trump       & nt          & us            & assets       \\
kilmeade          & renminbi          & white       & privilege   & texas         & house        \\
vox               & haram             & illegal     & alien       & boko          & haram        \\
chomsky           & bangkok           & like        & college     & emerging      & markets      \\
courteous         & checker           & law         & shall       & dan           & patrick      \\
gofundme          & nigerians         & person      & shall       & southeast     & asian        \\
rcp               & newsom            & legislature & may         & direct        & investment   \\
newsmax           & utaustin          & without     & warning     & sovereign     & wealth       \\
anarchism         & heremore          & threats     & violence    & us            & current      \\
trolling          & tribune           & monday      & friday      & guest         & post         \\
cavuto            & cfr               & agree       & terms       & members       & may          \\
fucking           & custodial         & obama       & nt          & us            & exports      \\
foxnewscom        & myanmar           & us          & maintain    & china         & trade        \\
anarchist         & rakhine           & news        & hour        & growth        & china        \\
susteren          & hun               & bill        & oreilly     & us            & firms        \\
grahamcassidy     & grist             & obamacare   & repeal      & net           & exports      \\
newsletter        & thai              & media       & matters     & research      & associate    \\
chez              & exporters         & news        & team        & exchange      & rates        \\
usmc              & nigeria           & black       & panther     & international & institutions \\
splc              & denuclearization  & hate        & group       & travis        & county       \\
omalley           & crossposted       & independent & journalism  & global        & governance   \\
fuck              & anc               & van         & susteren    & private       & investors    \\
banter            & yen               & false       & flag        & balance       & sheet        \\
madsen            & depreciation      & season      & two         & story         & updated      \\
watters           & rebalance         & research    & team        & jacob         & zuma         \\
beyoncé          & schwarzenegger    & happening   & world       & china         & central      \\
willard           & thailand          & corporate   & media       & china         & government   \\
jerk              & aggregator        & romney      & leads       & east          & north        \\
lgbtq             & sponsors          & officer     & darren      & texas         & austin       \\
globalist         & kyoto             & privately   & owned       & development   & goals        \\
odonnell          & lima              & overdose    & deaths      & suu           & kyi          \\
emmys             & pri               & shall       & made        & think         & worth        \\
mises             & israelpalestinian & darren      & wilson      & advanced      & economies    \\
globalists        & outflow           & big         & league      & bretton       & woods        \\
kliff             & suu               & show        & today       & president     & jacob        \\
individualist     & nigerian          & associate   & editor      & news          & views        \\
fck               & nld               & game        & thrones     & north         & texas        \\
machado           & uschina           & author      & necessarily & texas         & senate       \\
anarchists        & cyberspace        & basic       & income      & states        & china        \\
painkillers       & rebalancing       & ruling      & class       & david         & dewhurst     \\
slager            & inaudible         & divestment  & sanctions   & chinese       & state        \\
shall             & ph                & support     & continue    & think         & china        \\
zionists          & odinga            & america     & health      & southeast     & asia         \\
teabaggers        & bretton           & mr          & comey       & fort          & worth        \\
shep              & hu                & romney      & tax         & african       & national     \\ \bottomrule
\end{tabular}
\caption{Most relevant hyperpartisan and non-hyperpartisan words and bigrams according to log-odd ratios.}
\label{tab:words_bigrams}
\end{table*}

Relevant words of hyperpartisan articles:

\begin{itemize}
\item
  Hyperpartisan articles contain words ending in
  -ist/-ism/-ity(anarchist, anarchism, globalist, globalists,
  individualist, anarchists, zionists, vulgarity, profanity). These do
  not appear in non-hyperpartisan words.
\item
  Other hyperpartisan words that describe people (slager, teabagger,
  shep, lgbtq, courteous). Similar terms do not appear in
  non-hyperpartisan words.
\item
  Bad words in hyperpartisan articles (fucking, trolling, fuck, fck).
  There are no bad words in non-hyperpartisan.
\item
  News sites or webs in hyperpartisan articles (wonkette,
  realclearpolitics, newsbusters, vox, gofundme, newsmax, foxnewscom).
  This suggest that these news sites are commonly associated with
  hyperpartisan news. Most correspond to news agencies in the US. No
  news agencies appear in non-hyperpartisan words.
\item
  Other organizations in hyperpartisan news (usmc (United States Marine
  Corps), splc (Southern Poverty Law Center), emmys). They are US
  organizations.
\item
  People in hyperpartisan articles (oreilly, kilmeade, chomsky, cavuto,
  grahamcassidy, omalley, madsen, beyoncé, willard, odonell, kliff,
  machado, watters, susteren). They correspond to politicians,
  journalists and famous people.
\end{itemize}

Relevant words of non-hyperpartisan articles:

\begin{itemize}
\item
  Demonyms in non-hyperpartisan articles (subsaharan, nigerians, thai,
  israelpalestinian, nigerian). They correspond to people from other
  countries. No demonyms appear in hyperpartisan words.
\item
  Places in non-hyperpartisan articles (bangkok, myanmar, rakhine,
  nigeria, thailand, tribune, kyoto, lima). Many places appear in
  non-hyperpartsan words, none in hyperpartisan words. They correspond
  to other countries and cities.
\item
  People in non-hyperpartisan articles (straus, zuma, newsom, hun,
  schwarzenegger, hu (Hu Jintao)). They correspond to politicians and
  famous people.
\item
  Organizations in non-hyperpartisan articles (treasuries, boko, haram,
  tic, utaustin (The University of Texas at Austin), anc (African
  National Congress, pri (Partido Revolucionario Institucional), nld
  (National League for Democracy)). Unlike hyperpartisan organizations,
  many organizations are from other countries different from the US.
\item
  There are many economics terms in non-hyperpartisan articles
  (renminbi, yen, rebalance, cfr, depreciation, exporters, aggregator,
  outflow). Some correspond to currencies and other to actions or
  people.
\end{itemize}

\subsection{Relevant Bigrams}

There are many differences betweeen the 50 most relevant bigrams of
hyperpartisan and non-hyperpartisan news. Here are the main findings of
each class.

Relevant bigrams of hyperpartisan articles:

\begin{itemize}
\item
  More negative words than on non-hyperpartisan news (threats violence,
  hate group, divestment sanctions, overdose deaths, illegal alien)
\item
  People (obama, trump, bill oreilly, romney, darren wilson, mr comey,
  van susteren). They correspond to politicians or famous people
\item
  Media related terms (media research, independent journalism, media
  matters, corporate media, associate editor)
\item
  Politics related terms (trump obama, legislature, obamacare, america
  health, basic income, ruling class)
\end{itemize}

Relevant bigrams of non-hyperpartisan articles:

\begin{itemize}
\item
  Demonyms in non-hyperpartisan articles (southeast asian, african).
\item
  Places in non-hyperpartisan news (texas, us, china, southeast asia,
  travis county, austin). Some places are repeated a lot in different
  bigrams: china, us and texas are the most repeated ones.
\item
  Many economics related bigrams also appear a lot (emerging economies,
  exchange rate, emerging markets, direct investment, private
  investors\ldots).
\item
  Organizations (boko haram, international institutions, china
  gorvernment\ldots)
\item
  People are also mentioned (dan patrick, jacob zuma, suu kyi, president
  jacob, david dewhurst)
\end{itemize}

\subsection{Scattertext}

\section{Conclusions}

\bibliography{hyperpartisan_news_detection}
\bibliographystyle{acl_natbib}

\end{document}
